%
% relazione_es2.tex
%
% Copyright (C) 2016 frnmst (Franco Masotti) <franco.masotti@student.unife.it>
%                    dannylessio (Danny Lessio)
%
% This file is part of networks-lab.
%
% networks-lab is free software: you can redistribute it and/or modify
% it under the terms of the GNU General Public License as published by
% the Free Software Foundation, either version 3 of the License, or
% (at your option) any later version.
%
% networks-lab is distributed in the hope that it will be useful,
% but WITHOUT ANY WARRANTY; without even the implied warranty of
% MERCHANTABILITY or FITNESS FOR A PARTICULAR PURPOSE.  See the
% GNU General Public License for more details.
%
% You should have received a copy of the GNU General Public License
% along with networks-lab.  If not, see <http://www.gnu.org/licenses/>.
%


\makeatletter
\def\blfootnote{\xdef\@thefnmark{}\@footnotetext}
\makeatother

\documentclass[9pt, a4paper, oneside]{article}
\usepackage[a4paper]{geometry}
\usepackage{setspace} % packets
%\usepackage{verbatim} % include source files
\usepackage{amsmath} %powers
\usepackage{listings}
\singlespacing % interlinea singolo
\title{Relazione esercitazione 3 Laboratorio di reti\newline Server DNS con BIND e backend con OpenLDAP}
\author{Franco Masotti \and Danny Lessio}
\date{April 12, 2015}
\begin{document}
	\maketitle
	\tableofcontents
	\newpage
    \footnotetext{networks-lab  Copyright (C) 2016  frnmst (Franco Masotti), 
dannylessio (Danny Lessio).  This document comes with ABSOLUTELY NO WARRANTY.
This is free software, and you are welcome to redistribute it 
under certain conditions; see LICENSE file for details.}
	\part{Configurazione ed installazione del server DNS}
		\section{Ambiente di lavoro}
			\par
				Per la realizzazione di questo progetto ci 
				siamo basati su repository di
				\texttt{Arch Linux}, nei quali abbiamo 
				trovato i due pacchetti fondamentali per 
				implementare ed interrogare correttamente il 
				server DNS:
				\begin{itemize}
				\item
					Il pacchetto \texttt{bind}, ci fornisce 
					un server DNS configurabile (grazie 
					all'eseguibile \texttt{named}).
				 \item
					\texttt{dnsutils} ci fornisce un set 
					di comandi per poter interrogare il 
					server, come il comando 
					\texttt{dig} ed il comando 
					\texttt{host}.
				\end{itemize}
				Per l'installazione dei pacchetti \'e quindi 
				sufficiente fare:
				\begin{verbatim}
sudo pacman -Sy dnsutils bind
				\end{verbatim}
				Per maggiori informazioni riguardo il 
				funzionamento di entrambi ci siamo basati sia 
				sulla guida ufficiale
				\footnote{http://www.isc.org/software/bind/}
				, sia su altre guide come (soprattutto) quella
				di \texttt{Gentoo Linux}
				\footnote{https://wiki.gentoo.org/wiki/BIND/Tutorial}
				. Per quanto riguarda configurazioni del client 
				ci siamo basati sul wiki \texttt{Arch Linux}
				\footnote{https://wiki.archlinux.org/index.php/BIND}
				.
		\section{Definizione delle sottoreti e degli indirizzi IP}
			\par
				Abbiamo definito la struttura della nostra rete 
				in questo modo:
				\begin{itemize}
					\item
						Indirizzamento IPv4
						\begin{itemize}
							\item
								La nostra 
								sottorete \'e 
								\texttt{192.168.2.0/24}.
								Questo 
								significa che 
								abbiamo 
								$2^{(32-24)}
								= 256 $
								indirizzi per 
								gli 
								host
							\item
								Considerando 
								che non \'e 
								possibile 
								utilizzare gli 
								indirizzi che 
								terminano con 
								\texttt{0}
								e \texttt{255} 
								(perch\'e 
								\texttt{192.168.2.0}
								rappresenta la 
								sottorete 
								mentre 
								\texttt{192.168.0.255}
								rappresenta 
								l'indirizzo di 
								broadcast) 
								rimangono a 
								disposizione 
								254 indirizzi 
								realmente 
								utilizzabili 
								per gli host.
							\item
								Come indirizzo 
								del server DNS, 
								per comodit\'a 
								abbiamo scelto 
								\texttt{192.168.2.1}
								in quanto primo 
								indirizzo 
								disponibile.
						\end{itemize}
					\item
						Indirizzamento IPv6
						\begin{itemize}
							\item
								Quasi tutti i 
								concetti
								espressi per 
								l'IPv4 possono 
								essere usati 
								per l'IPv6.
								Tuttavia IPv6 
								non ha il 
								concetto di 
								indirizzo di
								broadcast.
							\item
								La nostra 
								sottorete IPv6 
								\'e 
								\texttt{2002::/64}
								, quindi 
								abbiamo a 
								disposizione circa 
								$2^{(128-64)} 
								= 2^{64}$
								indirizzi, 
								escludendo 
								\texttt{2002::0}
								che identifica 
								la sottorete.
							\item
								L'indirizzo del 
								server DNS \'e, 
								sullo stesso 
								concetto 
								dell'IPv4,
								\texttt{2002::1}.
						\end{itemize}
				\end{itemize}
		\section{Definizione delle zone}
			\par
				Per effettuare correttamente tutte le ricerce 
				DNS abbiamo definito tre zone: una di 
				indirizzamento diretto e due di indirizzamento 
				inverso. Le zone dicono sia come tradurre da 
				un host name ad un indirizzo IP (e viceversa), 
				sia quali informazioni autoritative dare (ad 
				esempio qual \'e il nome del server dns, 
				server mail, informazioni testuali, ecc...) 
				oltre che informare sui vari tempi in cui 
				scadono le richieste. Le nostre tre zone sono:
				\begin{itemize}
					\item
						\texttt{gruppo2.labreti.it}
						\begin{itemize}
							\item
								Questa \'e la 
								zona diretta, 
								cio\'e quella 
								che serve a 
								tradurre da 
								hostname ad 
								indirizzo IPv4 
								ed IPv6. Ogni 
								host infatti 
								ha due entry.
								Ogni entry ha 
								una serie di 
								atributi. Nel 
								nostro caso i 
								pi\'u 
								importanti sono 
								il campo 
								\texttt{name} 
								che identifica 
								un'host (o 
								atri oggetti) 
								ed il campo
								\texttt{type} 
								che ci dice se 
								si tratta di un 
								host (tipo 
								\texttt{IN} 
								cio\'e 
								\texttt{\textbf{IN}ternet}
								), oppure di un 
								name server 
								(tipo 
								\texttt{NS}).
								Il campo 
								\texttt{class}
								invece ci 
								definisce se 
								si tratta di un 
								indirizzo IPv4 
								(classe 
								\texttt{A}) 
								oppure IPv6 
								(classe 
								\texttt{AAAA}).
							\item
								Il server DNS 
								ha quindi due 
								entry: una di 
								tipo 
								\texttt{NS} 
								e l'altra di 
								tipo 
								\texttt{IN}. 
						\end{itemize}
					\item
						\texttt{2.168.192.in-addr.arpa}
						\begin{itemize}
							\item
								Questa \'e la 
								zona inversa 
								IPv4. Ha lo 
								scopo di 
								tradurre un 
								indirizzo IPv4 
								nel 
								corrispondente 
								host name. Da 
								notare l'uso 
								della classe 
								\texttt{PTR} 
								cio\'e 
								\texttt{PoinTeR}
								in cui la 
								parte host 
								degli indirizi 
								\emph{punta} ad 
								un nome di 
								machhina.
						\end{itemize}
					\item
						\texttt{0.0.0.0.0.0.0.0.0.0.0.0.0.0.0.0.0.0.0.0.0.0.0.0.2.0.0.2.ip6.arpa}
						\begin{itemize}
							\item
								Questa \'e la 
								zona inversa 
								IPv6, cio\'e  
								traduce gli 
								indirizzi IPv6
								nei  
								corrispondenti
								host names.
								\'E 
								concettualmente 
								uguale alla 
								precedente.
						\end{itemize}
				\end{itemize}
		\section{Definizione degli host}
			\par
				Abbiamo definito quattro host di esempio ed il 
				name server stesso (\texttt{ns}) per testare il 
				corretto funzionamento del server DNS. Abbiamo  
				associato i seguenti indirizzi IP alle 
				relative macchine:
		                \begin{table}[h]
					\begin{center}
                        			\begin{tabular}{cccc}
                                			\hline
			                                \multicolumn{3}{c}{\textbf{Hosts}} \\
                        			        \cline{1-3}
Host name & Indirizzo IPv4 & Indirizzo IPv6 \\ \hline
ns	&192.168.2.1&	2002::1\\
vaio	&192.168.2.10&	2002::10\\
lenovo	&192.168.2.20&	2002::20\\
asus	&192.168.2.30&	2002::30\\
acer	&192.168.2.40&	2002::40\\
\hline
			                        \end{tabular}
					\end{center}
                		\end{table}
		\section{Test del server DNS - Queries}
			\par
				A questo punto possiamo avviare lo script 
				\texttt{setup\_dns.sh} in modo che crei i file 
				di zona, come indicato nei listati.
			\par
				Di seguito mostriamo l'esecuzione del comando 
				\texttt{dig} per verificare il corretto 
				funzionamento. I primi esempi mostrano una 
				risoluzione diretta (IPv4 ed IPv6) grazie alla 
				zona \texttt{gruppo2.labreti.it}, mentre gli  
				ultimi mostrano la risoluzione inversa 
				rispettivamente con la zona IPv4 
				\texttt{2.168.192.in-addr.arpa} e con quella 
				IPv6 
				\texttt{0.0.0.0.0.0.0.0.0.0.0.0.0.0.0.0.0.0.0.0.0.0.0.0.2.0.0.2.ip6.arpa}.
			\par
				In entrambi i casi eseguiremo prima un 
				semplice comando \texttt{dig}, poi 
				useremo l'opzione \texttt{+short} che permette 
				di stampare a video solamente il corpo della 
				sezione \texttt{ANSWER SECTION}. 
				Successivamente lo switch \texttt{+short} 
				sar\'a abilitato per evitare ridondanze.
				\begin{itemize}
					\item
						Esempi con risoluzione diretta
						\begin{itemize}
							\item
								\texttt{dig ns.gruppo2.labreti.it} , 
								in questo modo 
								otteniamo la 
								risoluzione 
								IPv4 del nome 
								associato 
								al server DNS 
								(il comando 
								\texttt{dig}, 
								se non 
								specificato 
								diversamente, 
								effettua 
								solamente la 
								risoluzione 
								IPv4 di default):
								\begin{verbatim}
; <<>> DiG 9.9.7 <<>> ns.gruppo2.labreti.it
;; global options: +cmd
;; Got answer:
;; ->>HEADER<<- opcode: QUERY, status: NOERROR, id: 54163
;; flags: qr aa rd ra; QUERY: 1, ANSWER: 1, AUTHORITY: 1, ADDITIONAL: 2

;; OPT PSEUDOSECTION:
; EDNS: version: 0, flags:; udp: 4096
;; QUESTION SECTION:
;ns.gruppo2.labreti.it.		IN	A

;; ANSWER SECTION:
ns.gruppo2.labreti.it.	172800	IN	A	192.168.2.1

;; AUTHORITY SECTION:
gruppo2.labreti.it.	172800	IN	NS	ns.gruppo2.labreti.it.

;; ADDITIONAL SECTION:
ns.gruppo2.labreti.it.	172800	IN	AAAA	2002::1

;; Query time: 0 msec
;; SERVER: 192.168.2.1#53(192.168.2.1)
;; WHEN: Wed Apr 08 14:22:02 CEST 2015
;; MSG SIZE  rcvd: 108
								\end{verbatim}
							\item
								\texttt{dig ns.gruppo2.labreti.it AAAA} , 
								Attraverso lo 
								switch 
								\texttt{AAAA} 
								otteniamo la 
								risoluzione 
								IPv6 del nome 
				 				associato al 
				 				server DNS:
								\begin{verbatim}
; <<>> DiG 9.9.7 <<>> ns.gruppo2.labreti.it AAAA
;; global options: +cmd
;; Got answer:
;; ->>HEADER<<- opcode: QUERY, status: NOERROR, id: 54852
;; flags: qr aa rd ra; QUERY: 1, ANSWER: 1, AUTHORITY: 1, ADDITIONAL: 2

;; OPT PSEUDOSECTION:
;; EDNS: version: 0, flags:; udp: 4096
;; QUESTION SECTION:
;ns.gruppo2.labreti.it.		IN	AAAA

;; ANSWER SECTION:
ns.gruppo2.labreti.it.	172800	IN	AAAA	2002::1

;; AUTHORITY SECTION:
gruppo2.labreti.it.	172800	IN	NS	ns.gruppo2.labreti.it.

;; ADDITIONAL SECTION:
ns.gruppo2.labreti.it.	172800	IN	A	192.168.2.1

;; Query time: 0 msec
;; SERVER: 192.168.2.1#53(192.168.2.1)
;; WHEN: Wed Apr 08 14:58:07 CEST 2015
;; MSG SIZE  rcvd: 108
								\end{verbatim}
							\item
								\texttt{dig asus.gruppo2.labreti.it +short} , 
								in questo modo 
								otteniamo la 
								risoluzione 
								IPv4 dell'host 
								\texttt{asus}, 
								da notare 
								l'utilizzo 
								dello switch 
							 	\texttt{+short} 
							 	descritto 
							 	all'inizio 
							 	del paragrafo:
								\begin{verbatim}
192.168.2.30
								\end{verbatim}
							\item
								\texttt{dig asus.gruppo2.labreti.it AAAA +short} ,
								in questo modo 
								otteniamo la 
								risoluzione 
								IPv6 dell'host 
								\texttt{asus}:
								\begin{verbatim}
2002::30		
								\end{verbatim}
							\item
								\texttt{dig lenovo.gruppo2.labreti.it +short} , 
								\begin{verbatim}
192.168.2.20
								\end{verbatim}
							\item
								\texttt{dig lenovo.gruppo2.labreti.it AAAA +short} , 
								\begin{verbatim}
2002::20
								\end{verbatim}
						\end{itemize}
					\item
						Esempi con risoluzione inversa 
						in IPv4
						\begin{itemize}
							\item
								\texttt{dig -x 192.168.2.20} , 
								in questo modo 
								otteniamo la 
								risoluzione 
								inversa del 
								nome host
								associato a 
								questo 
								indirizzo IPv4.
								Il nome 
								restituito \'e
								\texttt{lenovo}:
								\begin{verbatim}
; <<>> DiG 9.9.7 <<>> -x 192.168.2.20
;; global options: +cmd
;; Got answer:
;; ->>HEADER<<- opcode: QUERY, status: NOERROR, id: 42862
;; flags: qr aa rd ra; QUERY: 1, ANSWER: 1, AUTHORITY: 1, ADDITIONAL: 3

;; OPT PSEUDOSECTION:
;; EDNS: version: 0, flags:; udp: 4096

;; QUESTION SECTION:
;20.2.168.192.in-addr.arpa.	IN	PTR

;; ANSWER SECTION:
20.2.168.192.in-addr.arpa. 172800 IN	PTR	lenovo.gruppo2.labreti.it.

;; AUTHORITY SECTION:
2.168.192.in-addr.arpa.	172800	IN	NS	ns.gruppo2.labreti.it.

;; ADDITIONAL SECTION:
ns.gruppo2.labreti.it.	172800	IN	A	192.168.2.1
ns.gruppo2.labreti.it.	172800	IN	AAAA	2002::1

;; Query time: 0 msec
;; SERVER: 192.168.2.1#53(192.168.2.1)
;; WHEN: Wed Apr 08 14:47:04 CEST 2015
;; MSG SIZE  rcvd: 154

								\end{verbatim}
							\item
								\texttt{dig -x 192.168.2.10 +short} , 
								\begin{verbatim}
vaio.gruppo2.labreti.it.
								\end{verbatim}
							\item
								\texttt{dig -x 192.168.2.40 +short} , 
								\begin{verbatim}
acer.gruppo2.labreti.it.
								\end{verbatim}
						\end{itemize}
					\item
						Esempi di risoluzione 
						inversa in IPv6
						\begin{itemize}
							\item
								\texttt{dig -x 2002::1} , 
								otteniamo la 
								risoluzione 
								inversa di 
								questo 
								indirizzo  
								IPv6, 
								corrispondente 
								al name server 
								DNS, cio\'e
								\texttt{ns.gruppo2.labreti.it}
								\begin{verbatim}
; <<>> DiG 9.9.7 <<>> -x 2002::1
;; global options: +cmd
;; Got answer:
;; ->>HEADER<<- opcode: QUERY, status: NOERROR, id: 923
;; flags: qr aa rd ra; QUERY: 1, ANSWER: 1, AUTHORITY: 1, ADDITIONAL: 3

;; OPT PSEUDOSECTION:
; EDNS: version: 0, flags:; udp: 4096

;; QUESTION SECTION:
;1.0.0.0.0.0.0.0.0.0.0.0.0.0.0.0.0.0.0.0.0.0.0.0.0.0.0.0.2.0.0.2.ip6.arpa. IN PTR

;; ANSWER SECTION:
1.0.0.0.0.0.0.0.0.0.0.0.0.0.0.0.0.0.0.0.0.0.0.0.0.0.0.0.2.0.0.2.ip6.arpa. 172800 
IN PTR	ns.gruppo2.labreti.it.

;; AUTHORITY SECTION:
0.0.0.0.0.0.0.0.0.0.0.0.0.0.0.0.0.0.0.0.0.0.0.0.2.0.0.2.ip6.arpa. 172800 
IN NS ns.gruppo2.labreti.it.

;; ADDITIONAL SECTION:
ns.gruppo2.labreti.it.	172800	IN	A	192.168.2.1
ns.gruppo2.labreti.it.	172800	IN	AAAA	2002::1

;; Query time: 0 msec
;; SERVER: 192.168.2.1#53(192.168.2.1)
;; WHEN: Wed Apr 08 15:25:31 CEST 2015
;; MSG SIZE  rcvd: 194

								\end{verbatim}
							\item
								\texttt{dig -x 2002::10 +short} , 
								in questo modo 
								otteniamo la 
								risoluzione 
								inversa del 
								nome associato 
								a questo 
								indirizzo 
								IPv6, infatti 
								ci viene 
								restituito il 
								nome dell'host 
								associato 
								cio\'e
								\texttt{vaio}:
								\begin{verbatim}
vaio.gruppo2.labreti.it.
								\end{verbatim}
							\item
								\texttt{dig -x 2002::30 +short} , 
								\begin{verbatim}
asus.gruppo2.labreti.it.	
								\end{verbatim}
						\end{itemize}
				\end{itemize}
		\section{File di configurazione}
			\par			
				Abbiamo creato un file di 
				configurazione chiamato \texttt{domain.conf} in 
				cui andiamo a definire tutte le variabili 
				necessarie per il funzionamento del server 
				\texttt{DNS}. Qui abbiamo definito il nostro 
				dominio (\texttt{gruppo2.labreti.it}), le 
				informazioni IPv4 ed IPv6 della nostra rete e 
				tutti gli host. In questo modo \'e possibile 
				definire nuovi computer semplicemente cambiando 
				alcuni array.
		\section{Script}
			\par
				Lo script \texttt{setup\_dns.sh} ci consente di 
				caricare automaticamante i dati contenuti in 
				\texttt{domain.conf} e di creare 
				automaticamente le tre zone (sia sotto 
				forma di zone-file sia come file 
				\texttt{ldif} da usare con LDAP). Viene anche 
				creato il file di configurazione di 
				\texttt{named}. Il funzionamento dettagliato 
				con le varie opzioni \'e rimandato alla parte 
				dei listati.
			\par
				Gli altri script vengono chiamati direttamente 
				da \texttt{setup\_dns.sh} a seconda delle 
				operazioni indicate dall'utente.
		\newpage
	\part{LDAP Backend}
		\section{Ambiente di lavoro}
			\par
				Grazie ad LDAP \'e possibile salvare le 
				informazioni riguardanti le zone e gli host in 
				modo che \texttt{named} possa collegarsi 
				direttamente.
				Utilizzando LDAP \'e pi\'u semplice fare 
				manutenzione e gestire gli host soprattutto 
				quando sono in quantit\'a elevata.
			\par
				Per effettuare il collegamento del nostro 
				server DNS al database LDAP siamo ricorsi al 
				pacchetto\footnote{http://www.bind9-ldap.bayour.com/bind-sdb-ldap-1.0.tar.gz} 
				\texttt{bind-sdb}. Questo 
				permette di inserire all'interno di 
				\texttt{named.conf} una entry, per ogni 
				zona, contenente l'URI LDAP corrispondente.
		\section{Installazione}
			\par
				Utilizzando \texttt{Arch Linux} il pacchetto 
				\texttt{bind} non \'e precompilato per poter 
				utilizzare il backend e non esiste il 
				pacchetto \texttt{bind-sdb} (come su 
				\texttt{CentOS}, \texttt{fedora}, ecc...). 
				Siamo dunque stati costretti a scaricare il 
				sorgente dal sito 
				ufficiale\footnote{https://www.isc.org} 
				la versione 
				\texttt{9.9.7}\footnote{ftp://ftp.isc.org/isc/bind9/9.9.7/bind-9.9.7.tar.gz} 
				ed a ricompilare l'intero pacchetto 
				collegandogli manualmente l'estensione 
				\texttt{bind-sdb}.
			\par
				\'E necessario inoltre scaricare lo schema 
				\texttt{dnszone}\footnote{http://www.bind9-ldap.bayour.com/dnszone-schema.txt} 
				da inserire nella cartella degli schemi di 
				LDAP, senza dimenticarsi della direttiva 
				\texttt{include} in 
				\texttt{/etc/openldap/slapd.conf}. Con 
				questo schema \'e possibile definire le zone 
				e gli host, quindi risulta fondamentale.
		\section{Configurazione}
			\par
				Una volta verificata la corretta compilazione, 
				abbiamo lanciato lo script 
				\texttt{setup\_dns.sh} che genera e carica nel 
				database LDAP i file \texttt{ldif} necessari 
				per le interrogazioni DNS. Questi file 
				sfruttano lo schema del punto precedente
				il quale definisce object class ed attributi per 
				il nostro scopo (ad esempio vengono forniti 
				gli attributi \texttt{A} ed \texttt{AAAA} 
				necessari per l'inserimento di campi IPv4 ed 
				IPv6 rispettivamente).
			\newpage
		\section{Albero LDAP}
			\par
				Questa \'e la struttura di una parte 
				dell'albero LDAP dopo aver lanciato lo script.
				\begin{itemize}
					\item
						gruppo2.labreti.it
						\begin{itemize}
							\item
								HOST
								\begin{itemize}
									\item
										gruppo2.labreti.it
										\begin{itemize}
											\item
												@
											\item
												ns
											\item
												vaio
											\item
												lenovo
											\item
												asus
											\item
												acer
										\end{itemize}
									\item
										2.168.192.in-addr.arpa
										\begin{itemize}
											\item
												@
											\item
												1
											\item
												10
											\item
												20
											\item
												30
											\item
												40
										\end{itemize}			
									\item
										0.0.0.0.0.0.0.0.0.0.0.0.0.0.0.0.0.0.0.0.0.0.0.0.2.0.0.2.ip6.arpa
										\begin{itemize}
											\item
												@
											\item
												1.0.0.0
											\item
												0.1.0.0
											\item
												0.2.0.0
											\item
												0.3.0.0
											\item
												0.4.0.0
										\end{itemize}
								\end{itemize}
						\end{itemize}
				\end{itemize}
		\section{Problemi riscontrati}
			\par
				Nella precente configurazione, LDAP era stato 
				implementato con il protocollo \texttt{ldaps} 
				(LDAP over TLS/SSL). Abbiamo tentato 
				l'allacciamento DNS attraverso questa 
				configurazione ma non abbiamo ottenuto alcun 
				risultato utilizzando l'URI 
				\texttt{ldaps://\emph{indirizzo}/} in 
				\texttt{named.conf}. 
				Per questo motivo \'e necessario cambiare 
				\texttt{slapd.service} modificando l'opzione 
				\texttt{ExecStart} cos\'i:
				\begin{verbatim}
ExecStart=/usr/bin/slapd -u ldap -g ldap -h "ldap:/// ldaps:///"
				\end{verbatim}
				Aggiungendo \texttt{ldap:///}, \texttt{slapd} 
				rimane in ascolto anche sulla porta standard 
	 			oltre che su quella sicura. A questo punto 
	 			verifichiamo il corretto funzionamento di named 
	 			con la nuova configurazione attraverso questi 
	 			comandi:
	 			\begin{verbatim}
sudo systemctl daemon-reload
sudo systemctl restart slapd.service
sudo systemctl status slapd.service
sudo systemctl restart named.service
sudo systemctl status named.service
	 			\end{verbatim}
	 			e controlliamo l'assenza di errori.
	 		\par
	 			\'E anche possibile 
	 			abilitare il \texttt{tls} mediante l'opzione 
	 			\texttt{!x-tls} in coda all'URI ma anche 
	 			questa soluzione non ha prodotto risultati. 
	 			Navigando in rete, sembra che in molti abbiano 
	 			riscontrato il nostro stesso problema. Il 
	 			backend con \texttt{ldaps} esula dalle 
	 			richieste di progetto.
		\newpage
	\part{Listati}
		\par
			Per problemi di spazio siamo andati a capo dove 
			necessario. Per il corretto funzionamento, riferirsi 
			direttamente al codice sorgente.
		\section{\texttt{domain.conf}}
			\par
				File di configurazione per il server DNS, in 
				cui vengono anche definiti tutti gli host ed 
				alcune configurazioni per il backed LDAP.
			\texttt{\lstinputlisting[breaklines]{../source/domain.conf}}
			\newpage
		\section{\texttt{setup\_dns.sh}}
			\par
				Script per il setup di BIND. Qui si pu\'o 
				scegliere tra cinque opzioni:
				\begin{itemize}
					\item
						\texttt{./setup\_dns -az}
						\begin{itemize}
							\item
								Esegue 
								l'installazione 
								automatica 
								creando i file 
								di zona e 
								modificando 
								\texttt{/etc/named.conf}.
						\end{itemize}
					\item
						\texttt{./setup\_dns -al}
						\begin{itemize}
							\item
								Esegue 
								l'installazione 
								automatica 
								creando i file 
								\texttt{ldif} 
								e caricandoli 
								nel database 
								LDAP.
								Viene inoltre 
								modificando 
								\texttt{/etc/named.conf}
								in modo che 
								utilizzi il 
								server LDAP 
								come backend.
						\end{itemize}
					\item
						\texttt{./setup\_dns -aaz} ed 
						\texttt{./setup\_dns -aal}
						\begin{itemize}
							\item
								Come le due 
								operazioni 
								sopra ma 
								vengono 
								modificate le 
								impostazioni di 
								rete 
								automaticamente.
						\end{itemize}
					\item
						\texttt{./setup\_dns -d}
						\begin{itemize}
							\item
								Cancella la 
								configurazione 
								per 
								\texttt{/etc/resolvconf.conf}
								in modo che al 
								riavvio del 
								computer 
								vengano 
								ripristinate 
								le precedenti 
								impostazioni di 
								\texttt{/etc/resolv.conf}
						\end{itemize}
				\end{itemize}
			\texttt{\lstinputlisting{../source/setup_dns.sh}}
			\newpage
		\section{\texttt{chg\_nw\_settings.sh}}
			\par
				Script per l'assegnazione degli indirizzi di 
				rete (validi fino al riavvio della macchina) 
				sull'interfaccia cablata.
			\par
				Lo script \texttt{chg\_nw\_settings.sh} ci 
				permette di assegnare gli indirizzi di tipo 
				IPv4 ed IPv6, le sottoreti e l'indirizzo di 
				broadcast IPv4 all'interfaccia di rete 
				cablata. Questi indirizzi corrispondono agli 
				indirizzi su cui \texttt{named} ascolta le 
				richieste.
			\par
				Per quanto riguarda l'assegnazione di indirizzi 
				IPv6 abbiamo riscontrato problemi inizialmente 
				per la struttura stessa del protocollo di rete. 
				Anche se si assegna un indirizzo di questo 
				tipo ad un'interfaccia il computer aspetta la 
				\emph{portante} in modo da evitare collisioni 
				con altri computer. Quindi un modo di risolvere 
				il problema \'e quello di attaccare un cavo di 
				rete all'intefaccia.
			\texttt{\lstinputlisting{../source/chg_nw_settings.sh}}
			\newpage
		\section{\texttt{chg\_resolv\_settings.sh}}
			\par
				Script per cambiare il file 
				\texttt{/etc/resolvconf.conf} in modo che 
				la risuluzione dei nomi host avenga con il 
				nostro server DNS.
			\par
				Per fare funzionare le query su un client \'e 
				necessario cambiare alcune impostazioni di 
				rete in modo che le ricerche con 
				\texttt{dig} vengano fatte con il nostro 
				server (senza bisogno di esplicitarlo) \'e 
				necessario modificare il file 
				\texttt{/etc/resolv.conf}, mettendogli in 
				testa gli indirizzi del nostro server. Nel 
				nostro caso gli indirizzi sono 
				\texttt{192.168.2.1} e \texttt{2002::1}. 
				Per fare un modo di non esplicitare sempre il 
				dominio di ricerca accanto all'hostname (nel 
				nosto caso \texttt{gruppo2.labreti.it}), quando 
				usiamo il comando \texttt{host}, \'e necessario 
				aggiungere anche questo nel file. Spesso, 
				tuttavia, le informazioni contenute in 
				\texttt{/etc/resolv.conf} vengono sovrascritte 
				ad ogni riavvio, o addirittura ad ogni evento 
				provocato dai pi\'u disparati network manager. 
				Per questo motivo abbiamo deciso di agire su 
				\texttt{/etc/resolvconf.conf} che ci permette 
				di fare tutte le cose scritte precentemente  
				senza che le informazioni vengano sovrascritte. 
			\texttt{\lstinputlisting{../source/chg_resolv_settings.sh}}
			\newpage
		\section{Zona diretta zone file}
			\par
				Di seguito troviamo le tre zone sotto forma di 
				zone file.
			\texttt{\lstinputlisting[breaklines]{../source/gruppo2.labreti.it}}
			\newpage
		\section{Zona inversa IPv4 zone file}
			\texttt{\lstinputlisting[breaklines]{../source/2.168.192.in-addr.arpa}}
			\newpage
		\section{Zona inversa IPv6 zone file}
			\texttt{\lstinputlisting[breaklines]{../source/0.0.0.0.0.0.0.0.0.0.0.0.0.0.0.0.0.0.0.0.0.0.0.0.2.0.0.2.ip6.arpa.}}
			\newpage
		\section{\texttt{named.conf} per i file di zona}
			\par
				Nota bene: con la notazione \texttt{IN} 
				nell'inestazione delle zone si identifica la 
				classe di dati \texttt{INternet}
				Se omessa nella definizione \'e comunque 
				assunta di default.
			\texttt{\lstinputlisting[breaklines]{../source/named.conf.zones}}
			\newpage
		\section{Zona diretta LDAP}
			\par
				Di seguito troviamo le tre zone sotto forma di 
				file \texttt{ldif}.
			\texttt{\lstinputlisting[breaklines]{../source/gruppo2.labreti.it.zone.ldif}}
			\newpage
		\section{Zona inversa IPv4 LDAP}
			\texttt{\lstinputlisting[breaklines]{../source/2.168.192.in-addr.arpa.zone.ldif}}
			\newpage
		\section{Zona inversa IPv6 LDAP}
			\texttt{\lstinputlisting[breaklines]{../source/0.0.0.0.0.0.0.0.0.0.0.0.0.0.0.0.0.0.0.0.0.0.0.0.2.0.0.2.ip6.arpa.zone.ldif}}
			\newpage
		\section{\texttt{named.conf} per LDAP}
			\par
				Nota bene: L'URI ha il \texttt{dn} 
				(\texttt{Distinguished Name}) diverso per ogni 
				zona, identificata da \texttt{ZoneName}.
			\texttt{\lstinputlisting[breaklines]{../source/named.conf.ldap}}
			\newpage
		\section{\texttt{named.service} per \texttt{systemd}}
			\par
				Poich\'e abbiamo compilato il pacchetto 
				\texttt{bind} \'e necessario cambiare alcune 
				configurazioni per il sistema di init.
			\texttt{\lstinputlisting{../source/named.service}}
\end{document}

%'\texttt{\symbol{92}0}'.
