%
% relazione_es3.tex
%
% Copyright (C) 2016 frnmst (Franco Masotti) <franco.masotti@student.unife.it>
%                    dannylessio (Danny Lessio)
%
% This file is part of networks-lab.
%
% networks-lab is free software: you can redistribute it and/or modify
% it under the terms of the GNU General Public License as published by
% the Free Software Foundation, either version 3 of the License, or
% (at your option) any later version.
%
% networks-lab is distributed in the hope that it will be useful,
% but WITHOUT ANY WARRANTY; without even the implied warranty of
% MERCHANTABILITY or FITNESS FOR A PARTICULAR PURPOSE.  See the
% GNU General Public License for more details.
%
% You should have received a copy of the GNU General Public License
% along with networks-lab.  If not, see <http://www.gnu.org/licenses/>.
%


\makeatletter
\def\blfootnote{\xdef\@thefnmark{}\@footnotetext}
\makeatother

\documentclass[9pt, a4paper, oneside]{article}
\usepackage[a4paper]{geometry}
\usepackage{setspace} % packets
%\usepackage{verbatim} % include source files
\usepackage{amsmath} %powers
\usepackage{listings}
\singlespacing % interlinea singolo
\title{Relazione esercitazione 4 Laboratorio di reti\newline Server Mail con Postfix e Dovecot}
\author{Franco Masotti \and Danny Lessio}
\date{April 26, 2015}
\begin{document}
	\maketitle
	\tableofcontents
	\newpage
    \footnotetext{networks-lab  Copyright (C) 2016  frnmst (Franco Masotti), 
dannylessio (Danny Lessio).  This document comes with ABSOLUTELY NO WARRANTY.
This is free software, and you are welcome to redistribute it 
under certain conditions; see LICENSE file for details.}
	\part{Configurazione ed installazione del server mail}
		\section{Scopo dell'esercitazione}
			\par
				L'obbiettivo di questa esercitazione \'e quello 
				di realizzare un server mail completo sia dal 
				punto di vista delle funzionalit\'a sia da 
				quello della sicurezza.
		\section{Componenti}
			\par
				Tutti gli strumenti necessari sono presenti 
				sia nei repository ufficiali sia in quelli non 
				ufficiali.
				Di seguito sono elencati componenti e scopo:
				\begin{itemize}
					\item
						Server SMTP
					\begin{itemize}
						\item
							Il server SMTP ha lo 
							scopo di inviare la 
							posta.
						\item
							Per questo componente 
							abbiamo utilizzato 
							\texttt{Postfix}\footnote{http://www.postfix.org/}
							. 
					\end{itemize}
					\item
						Server IMAP
					\begin{itemize}
						\item
							Questo server ha lo 
							scopo di ricevere la 
							posta grazie alle 
							notifiche del server 
							SMTP.
						\item
							Il componente che 
							abbiamo usato \'e
							\texttt{Dovecot}\footnote{http://dovecot.org/}
					\end{itemize}
					\item
						Autenticazione degli utenti
					\begin{itemize}
						\item
							Abbiamo effettuato 
							l'autenticazione 
							degli utenti LDAP con 
							\texttt{PAM}\footnote{http://linux-pam.org}, 
							in 
							particolare con il 
							pacchetto 
							\texttt{nss-pam-ldapd}\footnote{https://www.archlinux.org/packages/community/x86\_64/nss-pam-ldapd/}
							. 
							Questo tipo di 
							autenticazione vale sia 
							per Dovecot sia per 
							Postfix (che sfrutta 
							proprio Dovecot per 
							questo).
					\end{itemize}
					\item
						Sicurezza
					\begin{itemize}
						\item
							Per controllare le mail 
							abbiamo usato 
							\texttt{ClamAv}\footnote{http://www.clamav.net/} 
							e 
							\texttt{Spamassassin}\footnote{http://spamassassin.apache.org}.
							Il primo serve a 
							verficare la presenza 
							di virus mentre il 
							secondo per avvertire 
							il destinatario che si 
							tratta di spam.
						\item
							Per mettere in 
							comunicazione questi 
							due programmi con 
							Postfix abbaimo usato 
							\texttt{Amavis}\footnote{http://www.ijs.si/software/amavisd/}
					\end{itemize}
					\item
						Mail aliases
					\begin{itemize}
						\item
							Per effettuare il mail 
							aliasing, cio\'e la 
							traduzione da un nome 
							mail alternativo al 
							nome reale abbiamo 
							utilizzato Postfix che 
							va ad attingere 
							direttamente al 
							database LDAP.
					\end{itemize}
					\item
						SSL/TLS
					\begin{itemize}
						\item
							Tutte le comunicazioni 
							sia in entrata, sia in 
							uscita, sono protette 
							con SSL/TLS sfruttando 
							i certificati di 
							OpenLDAP. Questo lo si 
							pu\'o vedere nei file 
							di configurazione di 
							Postfix e Dovecot. 
						\item
							Inoltre non \'e 
							possibile utilizzare 
							comunicazioni 
							\emph{non} 
							protette.
					\end{itemize}
				\end{itemize}
		\section{Postfix}
			\par
				Nel file \texttt{main.cf} si trovano le 
				impostazioni di base valide per tutti i socket 
				in ascolto, mentre in \texttt{master.cf} \'e 
				possibile definire nuovi socket e fare 
				l'override delle impostrazioni (anche 
				aggiungendone di nuove).
			\par
				\textbf{main.cf}:
				\begin{itemize}
					\item
						Abbiamo configurato Postfix in 
						modo che utilizzi le directory 
						di tipo \texttt{Maildir}. In 
						questo modo viene creata tale 
						directory nella home di ogni 
						utente nella quale viene 
						recapitata la posta.
					\item
						Abbiamo configurato dominio ed 
						hostname del server SMTP 
						rispettivamente con 
						\texttt{gruppo2.labreti.it} e 
						\texttt{mail.gruppo2.labreti.it}
					\item
						Grazie alla direttiva 
						\texttt{alias\_maps} 
						\'e possibile 
						utilizzare gli alias 
						mail salvati 
						nell'albero LDAP. Abbiamo 
						creato un file chiamato 
						\texttt{ldap-aliases.cf} in cui 
						\'e definito come recuperare il 
						nome originario.
					\item
						Per fare in modo che l'utente 
						venga notificato dell'arrivo di 
						nuova posta (grazie a Dovecot) 
						bisogna esplicitare il valore 
						di \texttt{mailbox\_command}
					\item
						Abbiamo anche abilitato 
						esplicitamente l'ascolto in 
						IPv4 ed IPv6.
				\end{itemize}
			\par
				\textbf{master.cf}:
				\begin{itemize}
					\item
						All'interno di questo file 
						abbiamo definito la porta 587 
						(vedi RFC 
						2476\footnote{http://www.ietf.org/rfc/rfc2476.txt})
						come porta in ascolto sicura. 
						Infatti SSL/TLS \'e abilitato 
						di default. Se non fosse cos\'i 
						basterebbe fare sniffing del 
						traffico per intercettare le 
						password di autenticazione 
						oltre che per leggere la 
						posta. Tutto questo \'e stato 
						possibile anche grazie ai 
						certificati precedentemente 
						generati per OpenLDAP.
					\item
						Per mettere in comunicazione 
						anti-virus e anti-spam abbiamo 
						aperto la porta 10024 (socket 
						di tipo unix) e la porta 10025 
						(socket di tipo inet). La prima 
						\'e utlizzata da Postfix per 
						mandare ad Amavis le email da 
						scansionare mentre la seconda 
						da Amavis per mandare indietro 
						a Postfix le email valide.
					\item
						Per autenticare gli utenti, in 
						modo che solo chi \'e 
						registrato con username e 
						password possa inviare email,
						piuttosto che dover installare 
						e configurare un nuovo 
						pacchetto abbiamo utilizzato 
						l'autenticazione con 
						\texttt{SASL} di Dovecot per 
						tutti gli utenti (sia rete 
						interna, sia rete esterne).
						Questo lo si vede con la 
						direttiva 
						\texttt{-o smtpd\_sasl\_type=dovecot}
						.
				\end{itemize}
			\par
				\textbf{ldap-aliases.cf}:
				\begin{itemize}
					\item
						Nell'albero LDAP abbiamo creato 
						un'organizational unit chiamata 
						\texttt{ALIAS}. Ogni utente \'e 
						definito attraverso il campo 
						\texttt{rfc822MailMember} che 
						corrisponde al campo 
						\texttt{uid} nell' 
						organizational unit 
						\texttt{PEOPLE}. Il campo 
						\texttt{cn} rappresenta 
						l'alias.
					\item
						La ricerca viene quindi fatta 
						sul sottoalbero \texttt{ALIAS} 
						con \newline
						\texttt{query\_filter = (\&(cn=\%s)(objectClass=nisMailAlias))}
						cio\'e viene cercato l'alias 
						con il cn corrispondente, e che 
						faccia parte della classe 
						\texttt{nisMailAlias} (usata 
						solo dagli alias). Il valore 
						ritornato \'e di tipo 
						\texttt{rfc822MailMember} 
						(vedi lda direttiva: 
						\texttt{result\_attribute = rfc822MailMember})
						.
				\end{itemize}
		\section{Dovecot}
			\par
				Dovecot contiene molti fie di configurazione ma 
				per semplicit\'a abbiamo raggruppato tutto in 
				\texttt{dovecot.conf} dopo aver eliminato 
				tutti i commenti.
			\par
				\textbf{dovecot.conf}
				\begin{itemize}
					\item
						Per leggere la posta 
						utilizziamo esclusivamente il 
						protocollo \texttt{IMAP} che 
						permette agli utenti la 
						mobilit\'a. Le mail (ed il 
						loro stato) sono salvate sul 
						server. Con il 
						protocollo \texttt{POP} al 
						conrario si impone 
						di scaricare la mail dal server 
						per poi cancellarle da questo e 
						comunque non fornisce 
						sincronizzazione. POP non \'e 
						adatto alla gestione delle 
						mail tra pi\'u dispositivi e 
						per questo motivo sta cadendo 
						in disuso.
					\item
						Anche per Dovecot va definito 
						il tipo di mailbox, che deve 
						essere dello stesso tipo di 
						Postfix (vedi la direttiva:
						\texttt{mail\_location = maildir:~/Maildir})
						.
					\item
						Anche in questo caso abbiamo  
						utilzzato le chiavi pubbliche 
						e private di OpenLDAP per 
						l'SSL/TLS di Dovecot con le 
						direttive \texttt{ssl\_cert} ed 
						\texttt{ssl\_key}.
					\item
						NSS permette di mappare gli 
						utenti LDAP e di simularli 
						come se fossero appartenenti al 
						sistema\footnote{si pu\'o 
						verificare con 
						\texttt{getnent passwd}}, 
						PAM invece ne permette solo
						l'autenticazione. Se un utente 
						si logga per la prima volta 
						allora viene creata la sua home 
						directory (dove verranno 
						salvate le mail).
						La creazione automatica delle 
						home \'e garantita dalla 
						direttiva 
						\texttt{args = session=yes 
						dovecot} e dal modulo 
						\texttt{pam\_mkhomedir.so} 
						presente nel file 
						\texttt{dovecot} di PAM
						.
					\item
						Come accennato precedentemente, 
						Dovecot si occupa anche 
						dell'autenticazione SMTP 
						attraverso SASL. Avermmo potuto 
						utilizzare l'implementazione di 
						SASL provvista nella libreria 
						\texttt{cyrus-sasl}\footnote{https://www.archlinux.org/packages/extra/x86\_64/cyrus-sasl/} 
						ma abbiamo 
						notato che Dovecot possiede 
						gi\'a la propria 
						implementazione di SASL, quindi 
						abbiamo utilizzato tale 
						sistema senza installare altri 
						pacchetti (vedi la direttiva 
						\texttt{service\_auth}).
				\end{itemize}
		\section{Amavis}
			\par
				Amavis \'e un'interfaccia fra il mail server 
				SMTP (Postfix) e i mail filters (ClamAV e 
				Spamassassin). Abbiamo installato prima ClamAV 
				(un anti-virus libero), poi abbiamo aggiornato 
				i suoi database con il comando 
				\texttt{freshclam}. Successivamente lo abbiamo 
				testato con il test virus 
				\texttt{EICAR}\footnote{http://www.eicar.org/86-0-Intended-use.html}
				. Infine abbiamo modificato amavis.conf in 
				modo da abilitare la scansione dei virus.
			\par
				Amavis \'e in grado di gestire la configurazine 
				di Spamassassin (da installare 
				separatamente poich\'e non \'e fornita 
				un'implementazione). Infatti all'interno di 
				amavis.conf si possono modificare alcuni 
				parametri di Spamassassin tra cui anche 
				l'efettiva attivazione del servizio.
			\par
				Amavis \'e comunque in grado di gestire 
				un'ampia variet\'a di antivirus quindi non \'e 
				limitato a ClamAV.
	\part{Test}
		\section{Verifica delle funzionalit\'a}
			\par
				Per testare gradualmente le nostre 
				configurazioni abbiamo usato telnet, OpenSSL e 
				Sylpheed\footnote{http://sylpheed.sraoss.jp/en/} 
				(un client di posta grafico). Abbiamo 
				effettuato tre tipi di test finali:
				\begin{itemize}
					\item
						Mail normale:
						\begin{verbatim}
apr 23 18:00:50 antergos amavis[2455]: (02455-01) Passed CLEAN {RelayedInternal},
MYNETS LOCAL [192.168.2.1]:44188 
<dannylessio@mail.gruppo2.labreti.it> -> <jacktripper@mail.gruppo2.labreti.it>, 
Queue-ID: 9E0DB24270A, Message-ID: 
<20150423180030.8d49490ef0555c080a0b2e8c@mail.gruppo2.labreti.it>, mail_id: 
65zA1DuN5P5h, Hits: -1.01, size: 799, queued_as: E88A72426C3, 10260 ms
						\end{verbatim}
					\item
						Mail spam:
						\begin{verbatim}
apr 23 18:01:34 antergos amavis[2456]: (02456-01) Passed SPAMMY 
{RelayedTaggedInternal}, MYNETS LOCAL [192.168.2.1]:44193 
<jacktripper@mail.gruppo2.labreti.it> -> <dannylessio@mail.gruppo2.labreti.it>, 
Queue-ID: 51A7124270C, Message-ID: 
<20150423180124.6329dd42269b72556e2fb340@mail.gruppo2.labreti.it>, mail_id: 
FUf99b9scvM4, Hits: 3.453, size: 813, queued_as: 8DE512426E5, 10217 ms
						\end{verbatim}
					\item
						Mail contenente virus di prova:
						\begin{verbatim}
apr 23 18:02:38 antergos amavis[2455]: (02455-02) Blocked BANNED 
(application/x-msdownload,.dat,eicar.com) {DiscardedInternal,Quarantined}, 
MYNETS LOCAL [192.168.2.1]:44196 <dannylessio@mail.gruppo2.labreti.it> -> 
<jacktripper@mail.gruppo2.labreti.it>, quarantine: banned-wKIkH7wpTbg2, 
Queue-ID: 8C22A24270D, Message-ID: 
<20150423180238.207d7c28e55468544ea4f410@mail.gruppo2.labreti.it>, mail_id: 
wKIkH7wpTbg2, Hits: -, size: 1390, 79 ms		                                                                  
						\end{verbatim}
				\end{itemize}
		\newpage		
	\part{Listati}
		\section{\texttt{main.cf}}
			\par
				\texttt{/etc/postfix/main.cf}
			\par
				File di configurazione principale di Postfix.
			\texttt{\lstinputlisting[breaklines]{../source/etc/postfix/main.cf}}
			\newpage
		\section{\texttt{master.cf}}
			\par
				\texttt{/etc/postfix/master.cf}
			\par
				File di configurazione dei socket di Postfix. 
				Qui vengono inserite le opzioni di sicurezza e 
				di autenticazione.
			\texttt{\lstinputlisting[breaklines]{../source/etc/postfix/master.cf}}
			\newpage
		\section{\texttt{ldap-aliases.cf}}
			\par
				\texttt{/etc/postfix/ldap-aliases.cf}
			\par
				File di configurazione per il recupero del 
				nome originario (con LDAP) a partire 
				dall'alias.
			\texttt{\lstinputlisting[breaklines]{../source/etc/postfix/ldap-aliases.cf}}
			\newpage
		\section{File ldif con gli alias degli utenti}
			\par
				\texttt{gruppo2.labreti.it.aliases.ldif}
			\par
				Il file contiene gli alias degli utenti oltre 
				che l'ou ALIAS ed il record MX 
				(\emph{Mail eXchange server}) che indica 
				semplicemente qual \'e il server mail del 
				dominio.
			\texttt{\lstinputlisting[breaklines]{../source/gruppo2.labreti.it.aliases.ldif}}
			\newpage		
		\section{\texttt{dovecot.conf}}
			\par
				\texttt{/etc/dovecot/dovecot.conf}
			\par
				File di configurazione di Dovecot.
			\texttt{\lstinputlisting[breaklines]{../source/etc/dovecot/dovecot.conf}}
			\newpage
		\section{File di autenticazione di PAM per Dovecot}
			\par
				\texttt{/etc/pam.d/dovecot}
			\par
				File di configurazione per l'autenticazione 
				degli utenti di sistema ed LDAP. Da notare i 
				moduli \texttt{pam\_mkhomedir.so} per la 
				creazione automatica delle cartelle home e il 
				modulo \texttt{pam\_ldap.so} per il 
				collegamento al database LDAP.
			\texttt{\lstinputlisting[breaklines]{../source/etc/pam.d/dovecot}}
			\newpage
		\section{\texttt{nslcd.conf}}
			\par
				\texttt{/etc/nslcd.conf}
			\par
				File di configurazione di \texttt{nslcd} (local 
				LDAP name service daemon). Questo demone \'e 
				utilizzato per ottentere le informazioni degli 
				utenti dell'ou \texttt{PEOPLE}
			\texttt{\lstinputlisting[breaklines]{../source/etc/nslcd.conf}}
			\newpage
		\section{\texttt{nsswitch.conf}}
			\par
				\texttt{/etc/postfix/nsswitch.conf}
			\par
				File di configurazione di \texttt{NSS} (Name 
				Service Switch) utilizzato dal sistema per 
				definire dove si trovano i database di 
				amministrazione. Da notare le 
				direttive 
				\texttt{passwd}, \texttt{group}, 
				\texttt{shadow} che contengono la stringa 
				\texttt{ldap}.
			\texttt{\lstinputlisting[breaklines]{../source/etc/nsswitch.conf}}
			\newpage

		\section{\texttt{amavisd.conf}}
			\par
				\texttt{/etc/amavisd/amavisd.conf}
			\par
				File di configurazione di Amavis.
			\texttt{\lstinputlisting[breaklines]{../source/etc/amavisd/amavisd.conf}}
			\newpage
		\section{\texttt{clamd.conf}}
			\par
				\texttt{/etc/clamav/clamd.conf}
			\par
				File di configurazione principale 
				dell'antivirus ClamAV.
			\texttt{\lstinputlisting[breaklines]{../source/etc/clamav/clamd.conf}}
			\newpage
\end{document}
				
