%
% relazione_es7.tex
%
% Copyright (C) 2016 frnmst (Franco Masotti) <franco.masotti@student.unife.it>
%                    dannylessio (Danny Lessio)
%
% This file is part of networks-lab.
%
% networks-lab is free software: you can redistribute it and/or modify
% it under the terms of the GNU General Public License as published by
% the Free Software Foundation, either version 3 of the License, or
% (at your option) any later version.
%
% networks-lab is distributed in the hope that it will be useful,
% but WITHOUT ANY WARRANTY; without even the implied warranty of
% MERCHANTABILITY or FITNESS FOR A PARTICULAR PURPOSE.  See the
% GNU General Public License for more details.
%
% You should have received a copy of the GNU General Public License
% along with networks-lab.  If not, see <http://www.gnu.org/licenses/>.
%


\makeatletter
\def\blfootnote{\xdef\@thefnmark{}\@footnotetext}
\makeatother

\documentclass[9pt, a4paper, oneside]{article}
\usepackage[a4paper]{geometry}
\usepackage{setspace} % packets
\usepackage{listings}
\singlespacing % interlinea singolo
\title{Relazione esercitazione 8 Laboratorio di reti\newline
Server DHCP}
\author{Franco Masotti \and Danny Lessio}
\date{May 31, 2015}
\begin{document}
	\maketitle
	\tableofcontents
	\newpage
    \footnotetext{networks-lab  Copyright (C) 2016  frnmst (Franco Masotti),
dannylessio (Danny Lessio).  This document comes with ABSOLUTELY NO WARRANTY.
This is free software, and you are welcome to redistribute it
under certain conditions; see LICENSE file for details.}
	\part{Consegna}
			\par
				L'obbiettivo di questa esercitazione \'e quello 
				di installare e configurare un server 
				\texttt{DHCP}\footnote{Dynamic Host 
				Configuration Protocol.} con 
				i protocolli IPv4 ed IPv6. Bisogna abilitare il 
				\texttt{MAC}\footnote{Media Access Control.} 
				filterting in modo da permettere l'ottenimento 
				di un indirizzo IP solo a macchine 
				fidate\footnote{Questo vale esclusivamente per 
				IPv4.}.
		\newpage
	\part{Configurazioni del server DHCP}
		\section{Installazione del pacchetto}
			\par
				Il pacchetto che contiene il server DHCP 
				(sia v4 sia v6) \'e 
				\texttt{dhcp}
				\footnote{https://www.archlinux.org/packages/extra/x86\_64/dhcp/}
				. 
				Per IPv6 \'e necessario installare anche 
				\texttt{radvd}
				\footnote{https://www.archlinux.org/packages/extra/x86\_64/radvd/}
				.
		\section{Impostazione della rete}
			\par
				Prima di avviare i server \'e necessario 
				impostare correttamente gli indirizzi del 
				server. Abbiamo usato gli stessi indirizzi 
				delle precedenti esercitazioni quindi:
				\begin{itemize}
					\item
						Per IPv4: 
						\texttt{192.168.2.10/24}
					\item
						Per IPv6: \texttt{2002::a/64}
				\end{itemize}
			\par
				Per impostare gli indirizzi abbiamo usato il 
				network manager che consente di velocizzare le 
				operazioni. In questo modo le impostazioni 
				rimangono salvate anche dopo il riavvio senza 
				dover editare file.
		\section{Configurazione IPv4}
			\par
				Abbiamo creato una 
				subnet\footnote{\texttt{192.168.2.0}}
				definendo anche la maschera di 
				rete\footnote{\texttt{255.255.255.0}}.
				All'interno della subnet sono prsenti 
				informazioni sui server DNS, sul dominio, sui 
				router, ecc. Una impostazione fondamentale \'e 
				quella del range di indirizzi che gli host 
				possono ottenenere. Nel nostro caso vanno da 
				\texttt{192.168.2.11} a \texttt{192.168.2.254}.
			\par
				Ogni scheda di rete ha un indirizzo MAC 
				univoco. Si pu\'o quindi definire il MAC 
				filtering per host. Con la direttiva 
				\texttt{deny unknown-clients} 
				non \'e possibile l'ottenimento delle 
				informazioni della rete se l'host che le 
				richiede non \'e presente in un' opzione di 
				tipo \texttt{host} nel file del server DHCP. 
				Ci\'o viene fatto per evitare che computer
				sconosciuti non possano utilizzare la rete. 
				Questo sistema ha il difetto che se un 
				attaccante riesce ad ottenere un indirizzo MAC 
				valido allora pu\'o assegnare tale indirizzo 
				alla sua scheda di rete e quindi aggirare 
				questo sistema di sicurezza.\newpage
		\section{Configurazione IPv6}
			\par
				IPv6 richiede un componente in pi\'u rispetto 
				ad IPv4: il Router ADVertisment Daemon detto 
				\texttt{RADVD}. 
			\par
				Gli host mandano dei messaggi di tipo 
				\emph{router solicitation} per scoprire 	
				quali sono i router nella rete. RADVD risponde 
				con dei messaggi \emph{router adverisement}. 
				Questi ultimi contengono il prefisso 
				di rete oltre che l'indirizzo del router che 
				li ha mandati.
			\par
				Per implementare queste operazioni (impostare 
				una connessione di tipo stateful) dobbiamo 
				settare il file \texttt{/etc/radvd.conf}. 
				Abbiamo impostato questi parametri:
				\begin{verbatim}
AdvSendAdvert on;
AdvManagedFlag on;
AdvOtherConfigFlag on;
				\end{verbatim}
				Il primo parametro \texttt{AdvSendAdvert} 
				permette il RA (Router Advertisement).
				Gli altri due parametri 
				\texttt{AdvManagedFlag} e 
				\texttt{AdvOtherConfigFlag} permettono 
				l'implementazione del protocollo stateful. 				
		\section{Test}
			\par
				Per testare le configurazioni abbiamo usato due 
				computer (server e client) e abbiamo usato un 
				normale cavo ethernet per la connessione.
			\par
				A questo punto abbiamo testato DHCPv4 con il 
				seguente comando:
				\begin{verbatim}
sudo dhcpd
				\end{verbatim}
			\par
				Per lanciare DHCPv6 occorre quindi avviare 
				prima radvd:
				\begin{verbatim}
systemctl start radvd
				\end{verbatim}
				Successivamente occorre lanciare il comando 
				per far partire il server DHCP:
				\begin{verbatim}
/usr/bin/dhcpd -6 -cf /etc/dhcpd6.conf -lf /var/lib/dhcp/dhcpd6.leases enp4s0
				\end{verbatim}
				Il flag \texttt{-6} implica che dhcpd viene 
				avviato con il protocollo IPv6. Il flag 
				\texttt{-cf} impone \texttt{/etc/dhcpd6.conf} 
				come file di configurazione. Il flag 
				\texttt{-lf} imposta 
				\texttt{/var/lib/dhcp/dhcpd6.leases} come lease 
				file. L'ultimo parametro invece specifica 
				l'interfaccia di rete sulla quale il server deve 
				girare.
		\section{Considerazioni finali}
			\par
				Abbiamo verificato l'ottenimento di un 
				indirizzo all'interno del range per entrambi i 
				protocolli. Cambiando l'indirizzo MAC nel file 
				\texttt{dhcpd.conf} \'e stato verificato che il 
				client non \'e pi\'u stato in grado di ottenere 
				un'indirizzo IPv4.
			\par
				Per quanto riguarda IPv6 abbiamo 
				verificato l'ottenimento di indirizzi 
				corretti solo con client Debian. Tuttavia su 
				altri sistemi testati il prefisso rimane a 128 
				bit contro i 64 che ci aspettavamo.
		\newpage				
	\part{Listati}
		 \par
		 \section{\texttt{dhcpd.conf} (IPv4)}
			\par
				\texttt{/etc/dhcpd.conf}
			\par
			\texttt{\lstinputlisting[breaklines]{../source/dhcpd.conf}}
			\newpage
		\section{\texttt{dhcpd6.conf} (IPv6)}
			\par
			 	\texttt{/etc/dhcpd6.conf}
			\par
			\texttt{\lstinputlisting[breaklines]{../source/dhcpd6.conf}}
			\newpage
		\section{\texttt{radvd.conf} (IPv6)}
			\par
			 	\texttt{/etc/radvd.conf}
			\par
			\texttt{\lstinputlisting[breaklines]{../source/radvd.conf}}
\end{document}
