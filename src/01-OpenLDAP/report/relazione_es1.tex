%
% relazione_es1.tex
%
% Copyright (C) 2016 frnmst (Franco Masotti) <franco.masotti@student.unife.it>
%                    dannylessio (Danny Lessio)
%
% This file is part of networks-lab.
%
% networks-lab is free software: you can redistribute it and/or modify
% it under the terms of the GNU General Public License as published by
% the Free Software Foundation, either version 3 of the License, or
% (at your option) any later version.
%
% networks-lab is distributed in the hope that it will be useful,
% but WITHOUT ANY WARRANTY; without even the implied warranty of
% MERCHANTABILITY or FITNESS FOR A PARTICULAR PURPOSE.  See the
% GNU General Public License for more details.
%
% You should have received a copy of the GNU General Public License
% along with networks-lab.  If not, see <http://www.gnu.org/licenses/>.
%


\makeatletter
\def\blfootnote{\xdef\@thefnmark{}\@footnotetext}
\makeatother

\documentclass[11pt, a4paper, oneside]{article}
\usepackage{setspace} % packets
\usepackage{verbatim} % include source files
\singlespacing % interlinea singolo
\title{Relazione esercitazione 2 Laboratorio di reti Openldap}
\author{Franco Masotti \and Danny Lessio}
\date{March 23, 2015}
\begin{document}
	\maketitle
	\tableofcontents
	\newpage
    \footnotetext{networks-lab  Copyright (C) 2016  frnmst (Franco Masotti), 
dannylessio (Danny Lessio).  This document comes with ABSOLUTELY NO WARRANTY.
This is free software, and you are welcome to redistribute it 
under certain conditions; see LICENSE file for details.}
	\part{Scelte di progetto}
		\section{Ambiente di lavoro}
			\par
				\texttt{OpenLDAP} \'e presente nei repository 
				di tutte principali distribuzioni 
				\texttt{GNU/Linux}. Tuttavia abbiamo trovato 
				particolarmente utile la documentazione 
				presente sul wiki di \texttt{Arch Linux} in 
				quanto esaustiva ma non dispersiva. Per questo 
				motivo abbiamo provato \texttt{OpenLDAP} su 
				sistemi derivati da questo. Si pu\'o trovare 
				la wiki per l'installazione su:
				\begin{verbatim}
http://wiki.archlinux.org/index.php/OpenLDAP
				\end{verbatim}
				e la gestione degli utenti e dei gruppi su:
				\begin{verbatim}
http://wiki.archlinux.org/index.php/LDAP_authentication
				\end{verbatim}
		\section{Definizione del dominio e dell'utente amministratore}
			\par
				Per prima cosa abbiamo creato il nostro 
				dominio che corrisponde alla radice 
				dell'albero. Questo albero, chiamato anche
				\texttt{DIT} (\texttt{Directory Information 
				Tree}), contiene tutte le entry dei gruppi e
				degli utenti, quindi \'e il nostro contenitore 
				di informazioni.
				Ogni elemento dell'albero ha una chiave
				primaria che lo identifica. Questa viene
				chiamata \texttt{DN} (cio\'e
				\texttt{Distinguished Name}).
				Il \texttt{DN} per 
				la radice \'e \texttt{gruppo2.labreti.it} che
				in notazione \texttt{LDIF} corrisponde a:
				\begin{verbatim}
dn: dc=gruppo2,dc=labreti,dc=it
				\end{verbatim}
			\par
				Abbiamo poi creato l'utente amministratore 
				chiamato \texttt{root} che ha \texttt{DN} 
				uguale a:
				\begin{verbatim}
dn: cn=root,dn=gruppo2,dn=labreti,dn=it
				\end{verbatim}
				Attraverso l'attributo \texttt{roleOccupant}
				stabiliamo il dominio su cui agisce, cio\'e 
				\texttt{gruppo2.labreti.it}.
		\section{Gruppi e utenti}
			\par
				Successvamente abbiamo creato quattro 
				\texttt{OU} (\texttt{organizationalUnit}) che 
				rappresentano i quattro gruppi in cui 
				suddividere le entry aventi caratteristiche in 
				comune:
				\texttt{PEOPLE}, \texttt{GROUPS}, 
				\texttt{HOST}, \texttt{DHCP}.
				L'\texttt{OU} \texttt{PEOPLE} contiene 
				inizialmente le entry di quattro persone, 
				mentre le altre \texttt{OU} serviranno 
				successivamente.
			\par
				Per generare le password degli utenti viene 
				utilizzato il comando \texttt{slappasswd} che 
				genera l'hash (con algoritmo \texttt{SSHA}, se 
				non diversamente specificato) della password in input.
				Successivamente le password criptate vengono 
				copiate nel file \texttt{users.ldif} 
				e applicate come valore dell' attributo 
				\texttt{userPassword}, presente in ogni utente. 
				Se \'e gi\'a presente un database, pu\'o 
				essere usata un'entry
				\texttt{add: userPassword} con 
				\texttt{userPassword: <hashed user password>} 
				nella riga successiva. Infine \'e necessario 
				lanciare \texttt{ldapadd} oppure 
				\texttt{ldapmodify} a seconda dei casi.
			\par
				Abbiamo scelto la password ed il numero di 
				telefono come dati sensibili. Per definire chi 
				e come possa leggere questi dati, bisogna 
				aggiungere delle direttive in 
				\texttt{slapd.conf}. In questo modo per leggere 
				i dati sensibili \'e necessario autenticarsi 
				con gli switch \texttt{-W} e \texttt{-D}, 
				altrimenti, con \texttt{-x} e \texttt{-b} tali 
				dati vengono semplicemente omessi.
		\section{Script}
			\par
				Per automatizzare l'inizializzazione del
				database, il suo reset e per effettuare
				ricerche di prova abbiamo scritto alcuni script 
				shell. Questo ci ha permesso di risparmiare 
				parecchio tempo e di capire meglio il 
				funzionamento del sistema \texttt{LDAP}.
		\section{Object classes e schemas}
			\par
				Per scegliere gli attributi che ci 
				interessano, questi vanno scelti solo da 
				alcune object class. Infatti esistono tre tipi 
				di object class:
			\begin{itemize}
				\item
					\texttt{ABSTRACT}: classe \texttt{top} 
					che identifica la fine di una gerarchia 
					di classi. A questo tipo di classe, 
					quindi, corrisponde solo la classe 
					\texttt{top}.
				\item
					\texttt{AUXILIARY}: all'interno di 
					ogni entry possiamo inserirne in 
					numero arbitrario.
				\item
					\texttt{STRUCTURAL}: all'interno di 
					ogni entry possiamo inserirne sono una 
					che abbia come parent class (diretta) 
					la classe \texttt{ABSTRACT} 
					(cio\'e \texttt{top}), mentre è 
					possibile inserirne altre in cascata 
					che abbiano  come parent class una 
					specifica classe. Ad esempio:
					\begin{itemize}
						\item
							\texttt{inetOrgPerson}
							dipende da:
						\item
							\texttt{organizationalPerson}
							dipende da:
						\item
							\texttt{person} dipende 
							da:
						\item
							\texttt{top}
					\end{itemize}
			\end{itemize}
			\par
				Gli attributi e le object class sono definite 
				negli schemi (cio\'e nei file \texttt{.schema}
				in \texttt{/etc/openldap/schema}).
				Ogni object class pu\'o contenere pi\'u
				attributi e ne definisce le propriet\'a, tra 
				cui i vincoli \texttt{MUST} e \texttt{MAY}.
				Ogni attributo pu\'o essere contenuto in pi\'u
				object class.
		\section{LDAPS}
			\par
				Abbiamo creato uno script per la generazione 
				del certificato con \texttt{openssl}. In questo 
				modo tutte le operazioni di copia dei 
				certificati e di configurazione del server sono 
				automatizzate grazie a \texttt{make\_cert.sh}.
				Il \texttt{DN} da immettere \'e quello deciso 
				all'inizio, cio\'e \texttt{gruppo2.labreti.it}.
				Successivamente abbiamo usato due computer: uno 
				con il server e l'altro con il client di
				\texttt{openldap}. Prima di effettuare
				la query abbiamo modificato il file
				\texttt{/etc/openldap/ldap.conf} in modo che il 
				client possa accettare il nostro 
				certificato server che non \'e firmato da 
				\texttt{CAs} (\texttt{Certification 
				Authorities}). Abbiamo poi verificato la 
				connessione con TLS/SSL con la seguente query:
			\begin{verbatim}
ldapsearch -x -b "dc=gruppo2,dc=labreti,dc=it" \
	'(uid=*)'
			\end{verbatim}
				Questo ritorna tutti gli utenti presenti nel 
				database sul server all'indirizzo specificato 
				nella direttiva \texttt{URI} in 
				\texttt{/etc/openldap/ldap.conf}
			\par
				Per eventuali problemi da parte del client con 
				il certificato \'e sufficiente aggiungere una 
				variabile d'ambiente subito prima di lanciare 
				ldapsearch e/o specificare direttamente 
				l'\texttt{URI} con lo switch \texttt{-H}, in 
				questo modo:
				\begin{verbatim}
LDAPTLS_REQCERT=allow ldapsearch -H ldaps://192.168.0.2:636 -x -b \
	"dc=gruppo2,dc=labreti,dc=it" '(uid=*)'
				\end{verbatim}
			\par
				\'E inoltre possibile aggiungere lo switch
				\texttt{-d1} per vedere lo scambio di chiavi.
			\par
				Lanciando i comandi di ricerca con lo switch 
				\texttt{-Z}, per abilitare il \texttt{TLS}, 
				otteniamo un errore:
				\begin{verbatim}
ldap_start_tls: Operations error (1)
	additional info: TLS already started
				\end{verbatim}
				Tuttavia nelle impostazioni del client abbiamo 
				usato un \texttt{URI} con \texttt{ldaps} e con 
				la porta \texttt{636}. In questo modo il 
				servizio \texttt{TLS} \'e gi\'a definito. 
				Quindi 
				lo switch 
				\texttt{-Z} risulta ridondante perch\'e tenta 
				la duplice esecuzione di \texttt{TLS} e per 
				questo il programma avvisa dell'errore. Lo 
				switch, quindi, non va utilizzato.
			\newpage
	\part{Query di ricerca}
			\par
				\'E possibile effettuare le query di ricerca
				sia in locale sia in rete. Nel secondo caso \'e 
				sufficiente aggiungere lo switch \texttt{-H}
				che identifica il protocollo (\texttt{LDAPS}) e 
				l'host name o l'indirizzo ip del server.
				Ad esempio per effettuare una ricerca di 
				un'utente via rete \'e sufficiente installare 
				il pacchetto \texttt{openldap} su un secondo pc 
				che funger\'a da client. Successivamente si 
				aggiunge una riga al per l'accettazione del 
				certificato nel file di configurazione del 
				client. Infine si effettua la ricerca vera e 
				propria identificando la radice dell'albero 
				\texttt{DIT} come base di ricerca:
				\begin{verbatim}
sudo pacman -Sy openldap
sudo echo -e -n "TLS_REQCERT\tallow\n" >> /etc/openldap/ldap.conf
ldapsearch -H ldaps://192.168.0.2:636 -x '(uid=jacktripper)' -b \
	"dc=gruppo2,dc=labreti,dc=it"
				\end{verbatim}
				\begin{itemize}
					\item
						\texttt{-H} identifica il
						protocollo,  il server e la 
						porta di ascolto remota.
					\item
						\texttt{-x} significa di 
						utilizzare il metodo semplice 
						di autenticazione (anonimo).
					\item
						\texttt{-b} identifica la base 
						da cui incominciare la 
						ricerca.
				\end{itemize}
			\par
				Possiamo specificare i search pattern in questo 
				modo:
				\begin{verbatim}
ldapsearch -x '(|(uid=jacktripper)(cn~=Danny))'
				\end{verbatim}
				dove 
				\texttt{|} \'e l'operatore \texttt{OR}. In 
				questo modo vengono selezionati le entry aventi 
				\texttt{uid=jacktripper} con le entry aventi 
				\texttt{cn\~=Danny} cio\'e \texttt{cn} simile a 
				\texttt{Danny}. Gli operatori logici vengono 
				quindi messi per primi rispetto rispetto alle 
				coppie variabili valore.
			\par
				Con la seguente query otteniamo tutti gli 
				oggetti appartenenti all'\texttt{OU} people con 
				quelli apparteneti all'object class 
				\texttt{organizationalPerson}.
				\begin{verbatim}
ldapsearch -x -b 'dc=gruppo2,dc=labreti,dc=it' \
	'(&(ou=PEOPLE)(objectClass=organizationalPerson))'
				\end{verbatim}				
			\newpage
	\part{Listati}
		\section{slapd.conf}
			\par
				Questo file va posizionato in 
				\texttt{/etc/openldap/slapd.conf}. Si tratta 
				del file di configurazione del server. Le 
				direttive aggiunte si trovano in coda al file.
			\verbatiminput{../source/openldap_config/slapd.conf}
			\newpage
		\section{ldap.conf}
			\par
				File di configurazione del client posizionato 
				in \texttt{/etc/openldap/ldap.conf}. Da notare 
				la presenza di \texttt{TLS\_REQCERT     allow}
			\verbatiminput{../source/openldap_config/ldap.conf}
			\newpage
		\section{domain.ldif}
			\par
				Questo file \'e usato per la creazione del
				dominio e dei gruppi.
			\verbatiminput{../source/domain.ldif}
			\newpage
		\section{users.ldif}
			\par
				File usato per la creazione degli utenti.
			\verbatiminput{../source/users.ldif}
			\newpage
		\section{search\_examples.sh}
			\par
				Script per effettuare alcune query di ricerca.
			\verbatiminput{../source/search_examples.sh}
			\newpage
		\section{commands.sh}
			\par
				Script che chiama gli altri script per la 
				creazione del certificato, l'inizializzazione 
				con i dati di esempio e le query di esempio.
			\verbatiminput{../source/commands.sh}
			\newpage
		\section{make\_cert.sh}
			\par
				Script per la creazione  del certificato.
				L'amministratore si deve curare di fornire i
				dati corretti, sopratattutto il \texttt{DN},
				altrimenti l'autenticazione fallisce.
			\verbatiminput{../source/make_cert.sh}
			\newpage
		\section{restore\_config.sh}
			\par
				Script per il rispristino e la creazione della 
				configurazione iniziale (reset) del database.
			\verbatiminput{../source/restore_config.sh}
			\newpage
		\section{add\_example\_entries.sh}
			\par
				Script per l'aggiunta dei dati di esempio nel 
				database \texttt{LDAP} sfruttando 
				\texttt{domain.ldif} e \texttt{users.ldif}.
			\verbatiminput{../source/add_example_entries.sh}
			\newpage
		\section{slapd.service}
			\par
				Impostazioni di \texttt{systemd} per l'avvio di 
				\texttt{slapd}. Da notare l'utilizzo esclusivo 
				di \texttt{ldaps}.
			\verbatiminput{../source/openldap_config/slapd.service}
			\newpage
	
%\footnote{Per questioni di spazio riferirsi al codice sorgente per i commenti}:
%'\texttt{\symbol{92}0}'.

\end{document}
