%
% relazione_es4.tex
%
% Copyright (C) 2016 frnmst (Franco Masotti) <franco.masotti@student.unife.it>
%                    dannylessio (Danny Lessio)
%
% This file is part of networks-lab.
%
% networks-lab is free software: you can redistribute it and/or modify
% it under the terms of the GNU General Public License as published by
% the Free Software Foundation, either version 3 of the License, or
% (at your option) any later version.
%
% networks-lab is distributed in the hope that it will be useful,
% but WITHOUT ANY WARRANTY; without even the implied warranty of
% MERCHANTABILITY or FITNESS FOR A PARTICULAR PURPOSE.  See the
% GNU General Public License for more details.
%
% You should have received a copy of the GNU General Public License
% along with networks-lab.  If not, see <http://www.gnu.org/licenses/>.
%


\makeatletter
\def\blfootnote{\xdef\@thefnmark{}\@footnotetext}
\makeatother

\documentclass[9pt, a4paper, oneside]{article}
\usepackage[a4paper]{geometry}
\usepackage{setspace} % packets
\usepackage{listings}
\singlespacing % interlinea singolo
\title{Relazione esercitazione 5 Laboratorio di reti\newline
Configurazione di uno switch CISCO}
\author{Franco Masotti \and Danny Lessio}
\date{May 3, 2015}
\begin{document}
	\maketitle
	\tableofcontents
	\newpage
    \footnotetext{networks-lab  Copyright (C) 2016  frnmst (Franco Masotti), 
dannylessio (Danny Lessio).  This document comes with ABSOLUTELY NO WARRANTY.
This is free software, and you are welcome to redistribute it 
under certain conditions; see LICENSE file for details.}
	\part{Configurazione dello switch}
		\section{Consegna}
			\par
				L'obbiettivo di questa esercitazione \'e quello 
				di collegarsi ad uno switch CISCO e di 
				configurarlo in modo da creare delle 
				\texttt{VLAN}. Inoltre \'e richiesto di 
				collegare insieme pi\'u switch definendo 
				alcune delle porte come trunk.
		\section{Collegamento allo switch}
			\par
				Per collegarsi inizialmente allo switch \'e 
				necessario uno speciale adattatore dotato di 
				attacco seriale da attaccare alla porta 
				console presente nella parte posteriore 
				dell'apparato di rete. Poich\'e i computer  
				recenti non hanno un attacco seriale, \'e 
				necessario di un secondo adattatore da seriale 
				a USB.
			\par
				Una volta collegati tutti i cavi bisogna 
				individuare la periferica. Questo lo si pu\'o 
				fare con il classico comando \texttt{dmesg}. 
				Nel nostro caso la periferica \'e 
				\texttt{ttyUSB0}.
			\par
				A questo punto bisogna installare un programma 
				per mettere in comunicazione il computer con 
				lo switch. Noi abbiamo usato 
				\texttt{minicom}
				\footnote{https://alioth.debian.org/projects/minicom}
				, che \'e presente nei repository 
				ufficiali
				\footnote{https://www.archlinux.org/packages/?name=minicom}
				. Iniziamo la configurazione con 
				il comando
				\texttt{sudo minicom -s}\footnote{\'E
				necessario essere utente root per aprire la
				periferica \texttt{ttyUSB0}.}, dove \texttt{-s} 
				significa \emph{setup}. A questo punto 
				impostiamo i valori come indicato dalla 
				guida\footnote{https://help.ubuntu.com/
				community/CiscoConsole}. Se il collegamento va 
				a buon fine ci troviamo il terminale con la 
				scritta \texttt{Switch> }. A questo punto 
				inizia la configurazione vera e propria.
		\section{Configurazione di base}
			\par
				Prima di poter configurare qualunque cosa \'e 
				necessario entrare in modalit\'a 
				amministratore. Per farlo basta eseguire il 
				comando \texttt{enable}. A questo punto il 
				\emph{prompt} cambia e viene visualizzato 
				\texttt{Switch\# }\footnote{Simile a quando si 
				diventa utente \texttt{root} con \texttt{su} in 
				un terminale.}.
			\par
				Impostiamo la data e l'ora corente con 
				\texttt{clock set hh:mm:ss dd mmm\footnote{Il 
				mese pu\'o andare col nome completo (es: 
				\texttt{April}) oppure abbreviato (es: 
				\texttt{Apr})} yyyy}.
			\par
				In questa modalit\'a possiamo anche vedere 
				tutta la configurazione corrente con 
				\texttt{show running-config}.
			\par
				Adesso si entra in una modalit\'a ancora pi\'u 
				particolare con il comando \texttt{configure 
				terminal} dove abbiamo configurato i parametri 
				fondamentali. Il \emph{prompt} cambia 
				ancora, cos\'i: \texttt{Switch(config)\# }. 
				Qui si imposta il fuso orario con 
				\texttt{clock timezone GMT +1}. A questo punto 
				abbiamo cambiato l'hostname: \texttt{hostname 
				\emph{hostname}}\footnote{Nel nostro caso 
				\texttt{hostname gruppo2}.}, e cambiamo la 
				password di amministrazione con \texttt{enable 
				password \emph{newpassword}}\footnote{Nel 
				nostro caso \texttt{enable password labreti}. 
				Le nostre password, da questo punto saranno 
				tutte \texttt{labreti}.}. Abilitiamo ora il 
				servizio di cifratura delle password con 
				\texttt{service password-encryption}. 
				Infine controlliamo se la password risulta 
				criptata con \texttt{show running-config}.
			\par
				Se si vuole accedere allo switch da una 
				postazione remota bisogna abilitare 
				l'amministrazione. Questa si trova di default 
				nella VLAN 1. Rimaniamo nella modilit\'a 
				\texttt{configure terminal} e diamo in 
				sequenza i seguenti comandi:
				\begin{itemize}
					\item
						\texttt{interface 
						vlan1}\footnote{Seleziona 
						l'interfaccia.}
					\item
						\texttt{ip address 
						\emph{switch-address} 
						\emph{switch-subnet-mask}}
						\footnote{Nel nostro caso 
						\texttt{ip address 
						192.168.2.100 255.255.255.0}. 
						Ovviamente \'e necessario 
						cambiare la sottorete 
						dell'interfaccia cablata del 
						computer in modo che sia la 
						stessa di quella dello 
						switch.}
					\item
						\texttt{no 
						shut}\footnote{Abilita 
						l'interfaccia di rete.}
				\end{itemize}
				Collegiamo i due apparecchi con un cavo 
				ethernet e testiamo il tutto dal computer con 
				un \texttt{ping 
				\emph{switch-address}}\footnote{Nella nostra 
				configurazione con: \texttt{ping 
				192.168.2.100}}. Anche lo switch ha il 
				comando \texttt{ping}.
			\par
				Dopo aver configurato la rete si pu\'o 
				abilitare l'amministrazione con telnet (quindi 
				non c'\'e pi\'u bisogno del cavo seriale) per 
				controllare lo switch da una postazione remota. 
				Nonostrate telnet sia abilitato di default 
				questo non funzioner\'a correttamente finche 
				non impostiamo una password. \'E sufficiente 
				eseguire questi tre comandi, dopo essere 
				entrati nella modailt\'a \texttt{configure 
				terminal}:
				\begin{itemize}
					\item
						\texttt{line vty 0 
						4}\footnote{Abilita il login 
						fino a 5 sessioni 
						consecutive.}
					\item
						\texttt{login}
					\item
						\texttt{password 
						\emph{telnet-password}}
						\footnote{Setta la password 
						per il telnet. Nel nostro caso 
						\'e \texttt{labreti}}
				\end{itemize}
 		\section{Salvataggio e ripristino della configurazione}
			\par
				Dopo aver configurato la rete siamo pronti per 
				un salvataggio preliminare delle impostazioni. 
				Per salvare la configurazione dello switch 
				abbiamo usato un server tftp chiamato 
				\texttt{xinetd}\footnote{\texttt{xinetd} = 
				\emph{the extended Internet services daemon}.} 
				presente nei repository ufficiali
				\footnote{https://www.archlinux.org/packages/?name=xinetd}
				. Qui \'e sufficiente cambiare una 
				riga\footnote{Questo verr\'a 
				spiegato nella parte dei listati} nel file di 
				configurazione ed avviare il 
				servizio\footnote{Con \texttt{systemd}: 
				\texttt{sudo systemctl start xinetd}.}. Creiamo 
				la cartella di destinazione del 
				server\footnote{\texttt{mkdir /var/tftpboot 
				\&\& chmod 666 /var/tftpboot} in modo che tutti 
				gli utenti abbiano permessi di lettura e 
				scrittura.} e creiamo anche il file di 
				destinazione\footnote{\texttt{touch 
				/var/tftpboot/running-config \&\& chmod 666 
				/var/tftpboot/running-config}} (che deve 
				esistere prima dell'invio del file stesso).
			\par
				Torniamo sullo switch in modalit\'a 
				\texttt{enable} e copiamo le impostazioni 
				cos\'i: \texttt{copy running-config tftp:} 
				avendo cura di specificare l'indirizzo del PC e 
				il nome del file sorgente e 
				destinazione\footnote{Sempre 
				\texttt{running-config}.}. Se non ci sono 
				errori viene indicato il numero di byte 
				copiati.
			\par
				Per verificare il salvataggio \'e sufficiente 
				fare \texttt{cat /var/tftpboot/running-config} 
				e confrontarlo con l'output di \texttt{show 
				running-config} presente nello switch.
			\par
				\'E anche possibile ripristinare e rendere 
				persistente le impostazioni sullo switch. Per 
				farlo basta copiare \texttt{running-config} nel 
				file \texttt{startup-config} all'interno 
				dell'\emph{nvram} dello switch\footnote{
				\texttt{copy tftp: nvram:/startup-config}} (che 
				\'e la memoria non volatile. Il file 
				\texttt{startup-config} infatti viene caricato 
				all'avvio.
		\section{VLAN}
			\par
				La seconda parte dell'esercitazione consiste 
				nella creazione di alcune \texttt{VLAN} 
				(Virtual LAN) in modo da collegarsi anche 
				agli switch degli altri gruppi. Usando le VLAN 
				infatti \'e possibile creare molte reti 
				virtualmente separate ma fisicamente collegate. 
				In questo modo si risparmia il numero dei 
				cablaggi e degli apparati di rete.
			\par
				Per identificare univocamente le VLAN vengono 
				usati numeri interi da 1 a \emph{vlan-max}. 
				Inoltre ad ogni VLAN pu\'o essere associato un 
				nome sotto forma di stringa. Quindi, per prima 
				cosa, abbiamo creato la VLAN 2 di nome 
				\texttt{vlan2}, appunto. Entriamo nella 
				modalit\'a \texttt{enable} poi:
				\begin{itemize}
					\item
						\texttt{vlan 
						database}\footnote{Entra nella 
						modalit\'a di configurazione 
						delle VLAN.}
					\item
						\texttt{vlan 2 name 
						vlan2}\footnote{Assegna il nome 
						\texttt{vlan2} alla VLAN 2.}
					\item
						\texttt{exit}\footnote{Per 
						applicare la configurazione e 
						uscire.}
				\end{itemize}
				Per controllare che la configurazione sia 
				andata a buon fine usiamo il comando 
				\texttt{show vlan}, il quale ci mostrer\'a lo 
				stato di tutte le VLAN.
			\par
				Lo switch in dotazione ha 12 porte, quindi 
				possiamo assegnare almeno due porte per ognuna 
				delle quattro VLAN. Ora noteremo che tutte le 
				interfaccie fisiche sono assegate alla VLAN 1: 
				dobbiamo assegnarne 3 alla VLAN 2. Entriamo 
				prima nella modalit\'a \texttt{configure 
				terminal} poi:
				\begin{itemize}
					\item
						\texttt{interface 
						Fa0/2}\footnote{Entra nella 
						configurazione 
						dell'interfaccia 2}
					\item
						\texttt{switchport access 
						vlan 2}\footnote{Assegna 
						l'interfaccia 2 alla VLAN 2}
					\item
						\texttt{exit}
				\end{itemize}
				Ripetiamo queste operazioni anche per le porte 
				3 e 4.
				Ora rendiamo la VLAN 2 amministrativa nello 
				stesso modo dell VLAN 1. A questo punto 
				l'interfaccia amministrativa della VLAN 1 si 
				disattiva automaticamente.
			\par
				Con lo stesso procedimento per creare la VLAN 
				2 (esclusa la parte di amministrazione) abbiamo 
				creato anche le VLAN 11 e 3 (Con due porte 
				assegnate ciascuna).
			\par
				Per collegarci fisicamente agli switch degli 
				altri gruppi abbiamo usato dei cavi di tipo 
				crossover\footnote{https://en.wikipedia.org/wiki/Ethernet\_crossover\_cable} 
				poich\'e sono apparati dello stesso tipo, 
				quindi \'e necessario invertire ricezione e 
				trasmissione.
		\section{Porte trunk}
			\par
				Per poter mettere in comunicazione stesse VLAN 
				che si trovano su switch diversi i pacchetti 
				vengono incapsulati in modo che abbiano 
				un'identificativo con il numero della VLAN. 
				Tutto questo si ottiene assegnando una/pi\'u 
				porta/e come trunk. 
			\par
				Trovandoci fisicamente adiacenti sia al gruppo 
				1 sia al gruppo 3, abbiamo configurato le porte 
				1 e 12 come porte trunk. Per impostare una 
				porta trunk entriamo nella modalit\'a 
				\texttt{configure terminal}, poi:
				\begin{itemize}
					\item
						\texttt{interface Fa0/1}
					\item
						\texttt{switchport mode trunk}
						\footnote{Abilita linterfaccia 
						in modalit\'a trunk.}
					\item
						\texttt{switchport trunk 
						allowed vlan add 2,3,11}
						\footnote{Aggiunge le 
						VLAN 2,3,11 alla porta trunk, 
						dove la VLAN 11 \'e quella del 
						gruppo 1 per evitare conflitti 
						con la VLAN 1 stessa.}
				\end{itemize}
				Ripetiamo lo stesso procedimento per la porta 
				12 (cio\'e con \texttt{Fa0/12}).
		\section{Prove}
			\par
				Per testare le nostre cofigurazioni abbiamo 
				settato gli indirizzi delle interfaccie di rete 
				dei computer di tutti i gruppi nella sottorete 
				\texttt{192.168.1.0/24} e con il comando 
				\texttt{ping} abbiamo verificato di poter 
				raggiungere tutti i computer della sottorete.
		\newpage	
	\part{Listati}					
		 \section{\texttt{startup-config}}
			\par
				\texttt{startup-config}
			\par
				File di configurazione principale dello switch 
				da mettere in \texttt{nvram:/}.
			\texttt{\lstinputlisting[breaklines]{../source/running-config}}
			\newpage
		 \section{\texttt{tftp}}
			\par
				\texttt{/etc/xinet.d/tftp}
			\par
				File di configurazione del server tftp. Abbiamo 
				semplicemente cambiato l'ultima riga da 
				\texttt{disable = yes} a \texttt{disable = no}, 
				in modo da abilitare il servizio.
			\texttt{\lstinputlisting[breaklines]{../source/tftp}}
\end{document}
				
